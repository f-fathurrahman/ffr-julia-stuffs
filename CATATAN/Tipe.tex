
\chapter{Sistem Tipe}

Sistem tipe pada Julia bersifat dinamik, namun juga dapat mengambil keuntungan dari
sistem tipe statik karena juga memungkinkan kita untuk mengindikasikan tipe dari
suatu variabel atau data.

Secara default, tipe data pada Julia dapat dihilangkan sehingga suatu variabel
dapat memiliki tipe apa saja.

\section{Deklarasi tipe}

Deklarasi tipe pada Julia dapat dilakukan dengan menggunakan operator ::.
Misalkan:

\begin{juliacode}
α::Float64
Ω::Complex64
N::Int64 = 10
a::Float64 = 34.1
\end{juliacode}

Ketika operator :: diberikan setelah suatu ekspresi, maka operator ::
dibaca sebagai \textit{is an instance} of yang digunakan sebagai penegasan suatu
ekspresi akan memiliki suatu tipe tertentu.

Deklarasi juga dapat disematkan pada definisi fungsi:

\begin{juliacode}
function sinc(x)::Float64
  if x == 0
    return 1
  end
  return sin(pi*x)/(pi*x)
end
\end{juliacode}

Tipe abstrak

Tipe abstrak tidak dapat diinstansiasi dan hanya berfungsi sebagai node pada
graf tipe, sehingga mendeskripsikan himpunan dari tipe-tipe konkrit yang saling
berkaitan: tipe konkrit yang merupakan turunan dari tipe abstrak ini.
Meskipun tidak memiliki instansiasi, tipe abstrak diperlukan karena tipe abstrak
merupakan tulang punggung dari sistem tipe, sebagai hierarki konseptual yang
membuat tipe sistem pada Julia lebih dari sekedar kumpulan dari implementasi obyek.

Sintaks:

\begin{juliacode}
abstract type NamaTipe end
abstract type NamaTipe <: TipeSuper end
\end{juliacode}

Tipe primitif

Tipe primitif adalah tipe konkrit yang memiliki data plain old bits.

Tipe komposit

Biasa disebut dengan record, struct, atau object pada bahasa pemrograman yang
lain.

\begin{juliacode}
struct MyCompositeType
  Ndata::Int64
  v::Array{Int64,2}
end
\end{juliacode}

Fungsi fieldnames() dapat digunakan untuk mengetahui field apa saja
yang ada pada suatu instans dari tipe komposit.

Objek komposit yang dideklarasikan dengan struct bersifat *immutable*: artinya
objek tersebut tidak dapat dimodifikasi setelah konstruksi.


Misc

Untuk mengkonversi dari suatu tipe ke tipe yang lain dapat digunakan
fungsi convert().

Fungsi typeof() dapat digunakan untuk mengetahui tipe suatu variabel:

\begin{jlconcode}
julia> typeof(3.14)
Float64

julia> typeof(11)
Int64

julia> typeof(2//5)
Rational{Int64}

julia> typeof( (1,4,2) )
Tuple{Int64,Int64,Int64}
\end{jlconcode}

Fungsi ``subtypes()`` dapat digunakan untuk mengetahui subtipe dari suatu tipe
dan ``supertype()`` digunakan untuk mengetahui supertipe dari suati tipe.
