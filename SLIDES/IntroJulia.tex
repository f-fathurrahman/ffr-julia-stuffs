\documentclass[bahasa,10pt]{beamer}

\setlength{\parskip}{\smallskipamount}
\setlength{\parindent}{0pt}

\setbeamersize{text margin left=5pt, text margin right=5pt}

\usepackage{amsmath}
\usepackage{amssymb}
\usepackage{braket}

\usepackage{minted}
\newminted{jlcon}{breaklines,fontsize=\footnotesize,texcomments=true}
\newminted{julia}{breaklines,fontsize=\footnotesize,texcomments=true}
\newminted{bash}{breaklines,fontsize=\footnotesize,texcomments=true}
\newminted{text}{breaklines,fontsize=\footnotesize,texcomments=true}

\definecolor{mintedbg}{rgb}{0.95,0.95,0.95}
\usepackage{mdframed}

\BeforeBeginEnvironment{minted}{\begin{mdframed}[backgroundcolor=mintedbg]}
\AfterEndEnvironment{minted}{\end{mdframed}}

\setcounter{secnumdepth}{3}
\setcounter{tocdepth}{3}

\makeatletter
 \newcommand\makebeamertitle{\frame{\maketitle}}%
 \AtBeginDocument{%
   \let\origtableofcontents=\tableofcontents
   \def\tableofcontents{\@ifnextchar[{\origtableofcontents}{\gobbletableofcontents}}
   \def\gobbletableofcontents#1{\origtableofcontents}
 }

\makeatother

\usepackage{babel}

\begin{document}


\title{Dasar Pemrograman Julia}
\author{Fadjar Fathurrahman}
\institute{
Program Studi Teknik Fisika \\
Divisi Komputasi Pusat Penelitian Nanosains dan Nanoteknologi \\
Institut Teknologi Bandung
}
\date{7 April 2018}

\frame{\titlepage}

\begin{frame}[fragile]
\frametitle{Pemrograman Julia}

Contoh sesi interaktif
\begin{jlconcode}
   _       _ _(_)_     |  A fresh approach to technical computing
  (_)     | (_) (_)    |  Documentation: https://docs.julialang.org
   _ _   _| |_  __ _   |  Type "?help" for help.
  | | | | | | |/ _` |  |
  | | |_| | | | (_| |  |  Version 0.6.2 (2017-12-13 18:08 UTC)
 _/ |\__'_|_|_|\__'_|  |  Official http://julialang.org/ release
|__/                   |  x86_64-pc-linux-gnu

julia> 1 + 3.1
4.1
\end{jlconcode}

Contoh script
\begin{juliacode}
function myfunc( a::Int64, α::Float64 )
    println("a + α = ", a + α)
end
\end{juliacode}

\end{frame}




\end{document}

